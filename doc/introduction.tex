\section{Introduction}
\subsection{Overview}

Tracter is a data-flow framework for signal processing.  It defines a
library of components that each typically do a small amount of
computation, but can become nodes in a directed graph where they work
together (although independently) to do something more useful.

Although tracter contains several implementations of basic algorithms,
it has developed into a wrapper for libraries of basic algorithms.

The components generally run serially, not in parallel; tracter does
not do concurrent data-flow in the sense of Kahn process networks
\citep{Lee1995}.  It is not a language either.  However, the
components do have some things in common with process networks.  For
instance, reads are blocking but writes are not.

Tracter distinguishes sources and sinks.

\begin{figure}[h]
  \centering
  \begin{tikzpicture}[node distance=2cm]
    \node[state] (source) {};
    \node[below] at (source.south) {Source};
    \node[state] (component) [right of=source] {};
    \node[below] at (component.south) {Component};
    \node[state] (sink) [right of=component] {};
    \node[below] at (sink.south) {Sink};
    \path[->]
      (source) edge (component)
      (component) edge (sink);
  \end{tikzpicture}
  \caption{A minimal tracter graph.}
  \label{fig:Graph}
\end{figure}

%\begin{tikzpicture}[start chain]
%  \node [draw,on chain] {A};
%  \node [draw,on chain] {B};
%  \node [draw,on chain] {C};
%\end{tikzpicture}

\subsection{API}

One of the packages I use to write this document is called TikZ.  TikZ
is an astounding piece of software; I had no idea it was possible to
do things like that in \TeX.  However, the reason I mention it here is
that I think it has a very nicely designed API.  Tikz has 3 distinct
levels:
\begin{description}
\item[TikZ] itself is a top level interface to PGF.
\item[PGF] is the actual language.
\item[The underlying driver] layer disconnects PGF from the means of
  rendering the graphics.
\end{description}
For me, tikz is an example of a really well thought out API.  It's
well documented too - just the table of contents is 13 pages.

Tracter aims to have this sort of hierarchical API.  I'm not sure it
really succeeds, but the idea is to have about 4 or 5 layers:
\begin{description}
\item[Executables] are the top layer.  Generally speaking it's
  possible to write an executable that has just one tracter component
  other than a source and sink and does something useful.
\item[Factories] define a graph of components that constitute some
  useful block.  A factory is actually a very thin layer; it just
  instantiates a graph of components, so the overhead after calling
  the factory is zero.
\item[Components] are the main level.  A component is a node in a
  directed graph of processing elements.  Components necessarily have
  inputs and outputs.  To use them you have to buy into the data-flow
  idea.
\item[Objects] in tracter are things that have a name and can hence
  receive parameters.  They don't necessarily take part in data flow
  operations.  Factories are objects, as are components.
\item[Algorithms] are the bottom layer; fundamental functions.  These
  are the sorts of things that BLAS and LAPACK provide, and also
  comprise DFTs, filters and the like.  They might even be written in
  C or FORTRAN; not necessarily C++.
\end{description}

The most important level in tracter is the component
level.\footnote{At the time of writing, this is the plugin level.  But
  plugin is a bad name; it implies dynamic loading, which is not what
  tracter is about.}  It is possible to write only components.
However, if a component is really useful, it tends to be desirable to
break it down into other objects, but objects that still require
parameters to be passed to them.

Tracter is really the wrong place to put fundamental functions.  Those
belong in other libraries because they are useful outside the scope of
tracter.

An example of the hierarchy is a frequency warp.  It is possible to
write a program that has a small number of components that calculate a
frequency warp; the input is a periodogram and the output is a file,
or cepstrum or the like.  In this sense, the warp is a component in
the data-flow.  The frequency warp component itself is really just a
fundamental function.  However, it has a lot of parameters, and can be
used in different situations; this is an object in tracter terms.
Ultimately, however, the frequency warp is a function with parameters
that may even be implemented in some third party library; IPP for
instance.

This idea of fundamental libraries is one key point of tracter,
completely independent from the data-flow part.  It brings together
many independent libraries and allows them to be used in the same
framework.  This is not a new thing; Octave does it, as does MPlayer.

Somewhere in there is an interface layer too.  The idea that you can
have alternative implementations of the same functionality.  Typically
this is a DFT that could be from Kiss FFT, FFTW or even IPP.

There is a programming methodology called ``Extreme Progamming'' (EP).
EP encompasses many different philosophies, but one of them is the
idea that you should implement only as much as you need.\footnote{In
  some sense, tracter is way too over-engineered.}  The way to apply
EP in tracter is to first write a technique as a component.  If the
component is useful, expand it into one or more objects that are
contained by a component.  If the core algorithms are useful, they can
be abstracted out as fundamental functions.

That said, most of the fundamental functions are already written and
exist in other libraries.  In this sense, tracter reduces to an
exercise in writing Gnu configure scripts.


%%% Local Variables: 
%%% mode: latex
%%% TeX-master: "tracter"
%%% TeX-PDF-mode: t
%%% End: 
